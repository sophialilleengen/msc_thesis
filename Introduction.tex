\section{Introduction}\label{sec:Intro}
Galaxies are complex structures consisting of stars, gas, dust and \ac{DM} held together through gravity. The have many different shapes, sizes and colors and are in constant change. We can observe galaxies from our Galaxy, the \ac{MW}, over nearby galaxies, where we can still resolve individual parts, to high-redshift galaxies, when the Universe was still very young. This range of galaxies gives insight on galaxy formation and evolution through cosmic times. In their similarities, we can constrain many physical laws about galaxies and the Universe. 

\subsection{Why do we want to have a potential}
The gravitational potential of galaxies is fundamental to understand the structure of baryonic and invisible matter. It sets the foundation on how matter moves. We can observe the movement of stars and gas and draw conclusions on the total existing matter. The potential influences many direct and indirect observables such as rotation curves and properties of the different components of a galaxy. Many empirical correlations for galaxies were found which rely on the mass (therefore potential) of a galaxy. 

\subsubsection{Dark matter}
Since until now we can measure \ac{DM} only indirect via its gravitational effet, it is important to measure the total mass and the potential of galaxies. \ac{DM} is a challenging but fascinating idea. We will now give a very quick overview on the discovery, most promising models, problems and alternatives. \\
\\\textbf{History of \ac{DM} discovery} In 1933, \citeauthor{Zwicky...DM...1933} observed the motions of galaxies in the Coma clusters and found a much higher velocity dispersion than what should be expected by visible matter after applying the virial theorem. He introduced the term "dunkle Materie" (German for dark matter) which described the invisible matter. Almost 40 years later, Rubin \textit{et al.} \citeyearpar{Rubin...DM...1970, Rubin...DM...1978, Rubin...DM...1980} measured rotation curves of first the Andromeda galaxy, our closest spiral galaxy, then of many other edge-on disk galaxies. The visible mass content would let the rotation curve decline towards higher radii but the rotation curves stayed constant over a large range following 
\begin{equation}\label{eq:circ_vel}
    \mathrm{v_{circ}(r)} = \sqrt{\frac{\mathrm{GM(r)}}{\mathrm{r}}} \sim \mathrm{const}
\end{equation}
indicating that there are spheroidal halos around galaxies build up from invisible matter. Other observational methods also rely on \ac{DM}, such as strong (e.g. \citep{Trick..stronglensing...2016}) and weak gravitational lensing \citep{Tyson...weaklensing...1990, Kaiser...weaklensing...1993}. \ac{DM} seems to only interact via gravitational forces but not with electromagnetic radiation and therefore cannot be observed by light. Unfortunately, up to now, there has not been a direct detection of \ac{DM} in any way which causes great challenges but also brings many opportunities of research.\\
\\\textbf{Cosmological aspects of \ac{DM}}
In the current standard model of cosmology, the \ac{LCDM}-model, the Universe is made up by dark energy ($\Lambda$) and matter. Recent measurements of the \ac{CMB} by the \citet{Planck...CMB...2018} found, that dark energy makes up the biggest amount of the energy density ($\Omega_\Lambda = 0.685$) and matter the rest ($\Omega_m = 0.315$), split up to $\Omega_b = 5\%$ baryonic matter and $\Omega_c = 26.5\%$ cold dark matter assuming a Hubble constant of H$_0$ =  \SI{67.27}{km.s^{-1}.Mpc^{-1}}. Therefore, \ac{DM} makes up around \SI{84}{\%} of the total matter in the Universe. \\
\\\textbf{Established \ac{DM} model - cold dark matter}
\ac{CDM} was first introduced by \cite{Davis....CDM...1985} through \textit{N}-body simulations. \ac{CDM} particles are long-lived and very massive (\SI{10}{GeV} to a few TeV). These particles decoupled very early in the beginning stages of the Universe, already before reionization, and therefore be nonrelativistic. Then, they clustered and merged and was built bottom-up in a hierarchical growth.  Relativistic particles would destroy small scale substructure which would lead to larger voids than we observe. Possible particle candidates are \ac{WIMPs} which are massive particles interacting only via the gravitational and the weak force. The predicted large scale structure predicted by \ac{CDM} simulations agrees extraordinary well with the observed clustering of galaxies. \\
\\\textbf{Problems in the current model}
Even though the big success in explaining many phenomenons, \ac{CDM} has some problems on especially smaller scales (< \SI{1}{Mpc}) when comparing the predictions of cosmological \ac{DMO} simulations to observations ( e.g., \cite{Bullock...LCDMprobs...2017}). 
\begin{itemize}
    \item The missing satellites problem: these simulations predict many more satellites of disk galaxies in the low-mass end than we observe \citep{Klypin...missingsatellites...1999, Moore...missingsatellites..1999}. This can be explained by the fact that low mass \ac{DM} halos are extremely insufficient in forming galaxies and go completely dark below a certain threshold mass. In recent simulations analyzed by \citet{Sawala...noCDMproblems...2016} including baryons and physical prescriptions, the number of satellites matched the observations.
    \item The cusp-core problem: in these simulations, the halo density profile has a cusp in the center (citations) while observations find flatter density profiles and cored centers \citep{Flores...cuspcoreprob...1994, Moore...cuspcoreprob...1994}. A possible solution is including baryonic matter in these simulations which could be the driver for a cored center. 
    \item The too-big-to-fail problem \citep{Boylan...toobigtoofail...2011}: In the \ac{DMO} simulations, a large population of \acp{DM} satellites are found with greater central masses than any of the \ac{MW}'s dwarf spheroidals. These subhalos seem to have failed forming galaxies while halos with lower mass were successful. It was first found for the \ac{MW} but the same problem occurs for Andromeda \citep{Tollerud...M31tbtf...2014}, other \ac{LG} galaxies \citep{Kirby...LGtbtf...2014} and in more isolated lower mass galaxies \citep{Ferrero...DGtbtf...2012, Papastergis...DGtbtf...2015, Papastergis...DGtbtf...2016}.
    \iffalse\item The planes of satellites problem: \fi
\end{itemize}

\textbf{\ac{CDM} alternatives} Many alternatives for \ac{CDM} were suggested and many of them are already ruled out. Some of the alternatives which still are considered are 
\begin{itemize}
    \item \ac{WDM}: These particles should have masses of around \SI{1}{keV}. The mass grows bottom up down to a characteristic mass scale, where below the free streaming of the particles prevents the halos to form and the \ac{DM} is distributed in a smooth background field instead \citep{Smith...WDM..2011, Schneider...WDM...2013}. This theory predicts less low mass \ac{DM} halos whose densities would be less cuspy in the centers due to higher thermal motions \citep{Bode...WDM...2001}.
    \item \ac{MoND}: \cite{Milgrom...MoND...1983} suggested the idea of a modified theory of Newtonian law of gravity which only has an effect in low accelerations. This theory explains flat rotation curves and how galaxies move in clusters. A big advantage would be the non necessity of a new mysterious dark particle. Nevertheless, there are phenomenons such as the Bullet cluster \citep{Clowe...Bullett...2006} which fit perfectly in the \ac{CDM} universe but have struggles with finding explanations in \ac{MoND}.
\end{itemize}

\subsubsection{Empirical correlations}
In galaxies, many characteristics are correlated. There correlations are usually found empirically by analyzing and combining observational results. Many of them include the mass of a galaxy so once we know the mass we also know about other properties of the galaxy. 
\begin{itemize}
    \item \textbf{\citet{Tully...Fisher...1977}} (TF) determined a relationship between the luminosity L of a spiral galaxy and its radial velocity (which is connected to the mass of a galaxy through Equation \ref{eq:circ_vel}):
    \begin{equation}
        \mathrm{L \propto (v_{circ, max})}^\beta \qquad \mathrm{with}\ \beta =  2.5 - 5
    \end{equation}
    For the radial velocity, they measured the Doppler-broadened 21-cm radio emission line of neutral hydrogen (see Section \ref{subsec:mass_est_ext}). 
    \item \textbf{\citet{Faber...Jackson...1976}} measured the central radial velocity dispersion $\sigma_0$ of elliptical galaxies and found the relation to the luminosity  
    \begin{equation}
        \mathrm{L} \propto \sigma_0^4,
    \end{equation}
    which is similar to the \acs{TF} relation of spiral galaxies. The derivation of this relation made simple assumptions such as a uniform mass distribution on the volume of radius \textit{R} and a constant \ac{ML} ratio for all galaxies and equal surface brightnesses. Therefore, there is a large scatter in the data around this relation.
    \item \textbf{The Fundamental Plane} offers a better empirical fit to the data of ellipticals but needs another parameter, the effective radius $r_e$. It combines radius and luminosity of a galaxy with its gravitational well. Two representations of the fit are \citep{Carroll...Ostlie..2006}:
    \begin{align}
        L &\propto \sigma_0^{2.65}r_e^{0.65} \\
        r_e &\propto \sigma_0^{1.24}I_e^{-0.82}
    \end{align}
    \iffalse\item M\_vir - N\_GC \fi
\end{itemize}

\iffalse
\subsubsection{Application}
MW \acp{GC} proper motions and dynamics (including action distribution and dynamical model of potentials): \cite{Vasiliev...GCdynsGaiaDR2...2018}\\
Modelling the \ac{MW}'s \ac{GC} system: \cite{Binney...GCsystem...2017}
\fi
\subsection{Dynamical modelling methods}
Since we are not able to measure every star in the \ac{MW} yet alone to resolve stars in external galaxies, we need to make models of the observations to extract information. Measuring motions, dynamical models are of special interest. Stars in the disk and in the halo of galaxies can be considered as collisionless tracers which makes these models possible. 
\begin{itemize}
    \item \textbf{Jeans modelling} \citep{Jeans.....1915}: The first velocity moment of the \ac{CBE} (Equation \ref{eq:CBE}) relates the velocity ellipsoid of stars at a given position in a galaxy to the gravitational forces and the spatial \ac{DF} of the stars. The advantages are that no assumption on the \ac{DF} is needed and the computation is fast so a lot of different models can be explored. The set of Jeans equations is not closed and therefore does not have a unique solution. Therefore, assumptions need to be made and the solution, if found, may give non-physical results for e.g. the \ac{DF}. 
    \item \textbf{Schwarzschild's orbital superposition approach} \citep{Schwarzschild...1979}: Dynamical models of triaxial galaxies can be made based on observed surface brightness distribution and observed kinematics. Given a potential, an orbit library over the full integral of motion space is constructed. The number/mass/light of stars on a specific orbit are described by a weight. These weighted orbits build up the stellar \ac{DF}. By comparing the surface brightness and kinematics of the model with the data the gravitational potential can be recovered. This method is mostly used in external galaxies \citep{Rix...Schwarzschild...1997, vdBosch...Schwarzschild...2008, Vasiliev...Schwarzschild...2013, Ling...Schwarzschild...2018}.
    \item \textbf{Action-based modelling:} Orbits in axisymmetric potentials can be modelled with \acp{DF} which arrange stars in 3d action space instead of 6d phase space \citep{Binney...actionbasedmodelling...2012, Bovy...actionbasedmodelling...2013}. The modelling is similar to the Schwarzschild approach. The differences are that it does not numerically integrate orbits but uses orbital actions and tori and instead of orbits weights, the analytic \acp{DF} are physically motivated and action-based. Since we need 6d phase space information to calculate actions it is mainly used in the \ac{MW} where we can resolve single stars. It is successfully applied to modelling the disk (e.g., \citealt{trick...ROADMAPPING...2016, Wilmathesis}) but also to model the stellar halo (see Section \ref{subsec:mass_est_MW}). 
\end{itemize}
\subsection{Milky Way mass estimates}\label{subsec:mass_est_MW}
There are many different approaches to measure the mass and the potential of the Galaxy. Due to our position within the \ac{MW}, some methods which give very good constraints on overall parameters such as e.g., rotation curves, of external galaxies (see Section \ref{subsec:mass_est_ext}) cannot be measured as easily. A big advantage is that we can resolve stellar positions and velocities with high precision, especially with \textit{Gaia} \citep{Gaia...mission...2016, GaiaDR2...overview...2018, GaiaDR...GCs...2018}, which is helpful in both Galactic archaeology and dynamical modelling. These are some of the kinematic and dynamical methods to measure the Galactic mass in the halo (their results are presented in Table \ref{tab:MW_mass_estimations}):
\begin{itemize}
    \iffalse\item Kinematics of nearby stars: \cite{Kuijken...LocalDMdens...1989, Bovy...LocalDMdens...2012} \fi
    \item \textbf{Stellar streams in phase space:} Cold stellar streams are remnants of disrupted \acp{GC} and are a byproduct of hierarchical galaxy formation. \citet{Johnston...MWstreams...1999} first found, that these streams contain information about the Galaxy's gravitational potential. They move on orbits aligned to the satellite's orbit \citep{Eyre...streamstheo...2011} and their phase-space distribution is predominantly affected from the gravitational potential \citep{Kupper...streams...2010, Kupper...streams...2012}. It is therefore possible to get a rather direct measurement of the local acceleration close to the stream. So far, dynamical models of four single stellar streams have been used to constrain the mass and shape of the \ac{DM} halo (the Sagittarius dwarf galaxy \citep{Law...sagstream...2010, Gibbons...sagstream...2014, Dierickx...sagstream..2017}, the Orphan stream \citep{Newberg...orphanstream..2010}, the GD-1 stream \citep{Koposov...GD1stream...2010, Bowden...GD1stream...2015, Malhan...GD1stream...2018}, and the tails of the Palomar 5 globular cluster \citep{Kupper...pal5stream...2015}). Since these streams measure local properties, they give a better constraint on the global potential by looking at a population of streams \citep{Bonaca...streamsinfo...2018}.
    
    \item \textbf{Stellar streams in action space:} Dynamics of tidal stream have the most simple form in action-angle space \citep{Tremaine...streamsactiontheory...1999, Helmi...streamsactionstheory...1999}. A deeper introduction to actions is given in Section \ref{sec:Dynamics}. Due to formerly high computing costs calculating actions numerically, this approach has been carried on larger scale out only recently after developing new, cheaper methods for action calculations (a review is given in \citet{Sanders...actionreview...2016}). \citet{Streams...Sanders...2014} uses a St\"ackel-fitting algorithm \citep{Sanders...Staeckel...2012} to generate probabilistic models of streams to constrain the Galactic potential. \citet{Streams...Bovy...2014} introduces a new, general method of action-angle calculation using an orbit-integration-based approximation. In \citet{Streams..GD1..Pal5...Bovy...2016}, this method is then for the first time applied individually and combined to the Palomar 5 and GD-1 streams.
    \item \textbf{\ac{GC} distribution:} Another tracer of the mass of the inner \ac{MW} halo (r $\le$ \SI{21}{kpc}) is the \ac{GC} distribution. With the second data release of \textit{Gaia} (DR2) distances and \acp{PM} are now available for 154 \acp{GC} \citep{Baumgardt...GCoverview...2019} respective 150 \acp{GC} \citep{Vasiliev...GCoverview...2018} which are nearly all known \ac{MW} \acp{GC}. 
    \citet{MWmass...GCmotions...Watkins...2018} use the kinematics of a subgroup \acp{GC} in this inner halo to constrain the Galaxy's mass. \citet{Posti...MWmassGCs...2018} make another approach with fitting an action-based \ac{DF} to the 6d phase space data of 75 \acp{GC} and the mass and shape of the \ac{DM} halo. A very similar approach is carried out in \citet{Vasiliev...GCoverview...2018} which mainly differs in assumptions on the \ac{NFW} halo. \acp{PM} are measured by other telescopes as well. \citet{Sohn...GCsHST..2018} use \ac{HST} \acp{PM} to derive the mass of the \ac{MW} with the same method as \citet{MWmass...GCmotions...Watkins...2018} but less \acp{GC}. 

    \item \textbf{Satellite dynamics:} To constrain the mass of the outer halo we can use the dynamics of satellite galaxies. The methods are similar to the ones used for \ac{GC} estimates. From position and velocity of the most distant dwarf Leo I (r $= 257.8_{{-35.1}}^{+16.8} $ kpc), \citet{GaiaDR...GCs...2018} provided a lower limit on the \ac{MW} mass. \citet{MWmass...sat...dyn} calculate in the hydrodynamical simulations \acp{DF} of specific energy and angular momentum, with given 6d phase space information, which vary according to the galaxy host mass, estimate this mass by a maximum likelihood and apply this method to the \ac{MW}.
\end{itemize}

An overview of the results is shown in Table \ref{tab:MW_mass_estimations}.
\begin{table}[htbp]
\captionsetup{format=plain}
    \centering
    \begin{tabular}{@{}llll@{}}
         \toprule
         Method& Virial mass [$10^{12} \mathrm{M}_\odot$] & Rotational velocity [km s$^{-1}$]&Reference  \\
         
         \midrule

         \bottomrule 
    \end{tabular}
    \caption{Mass estimation results of the \ac{MW} }
    \label{tab:MW_mass_estimations}
\end{table}


\subsection{Mass measurements of external galaxies}\label{subsec:mass_est_ext}
In external galaxies, we cannot resolve single stars but, dependent on mass, luminosity and distance of the galaxy, we can resolve objects in the halo such as \acp{GC} to observing a galaxy as point source on the sky. To get a constraint on their mass it is useful to measure the rotational velocity of the galaxy (see Equation \ref{eq:circ_vel}). Different techniques evolved over time and telescope resolution, from measuring one value of the velocity to having spectra for each observed pixel.
\begin{itemize}
    \item\textbf{1D: 21cm line} Hydrogen is the simplest yet most abundant atom in space. The 21 cm line of neutral hydrogen (HI line) is visible through photons which are emitted when relative spins change from parallel to antiparallel. HI is detectable in radio bands in most spirals and some ellipticals. Radial velocities can be measured from the Doppler shift of these emission lines which give us a measurement of the enclosed mass.
    \item\textbf{2D: slit along the major axis} With the slit of a spectrograph aligned along the major axis of a galaxy, it is possible to take stellar spectra. Stars moving towards the observer are blue-shifted, stars moving away are redshifted. From the Doppler shift one can derive the rotational velocity and therefore the mass.
    \item \textbf{3D: \ac{IFU}} \ac{IFU} spectrographs observe the 2d field of view and take spectrums for each pixel at the same time. With that method we gain a 2d velocity map and can learn more about the mass distribution of the observed galaxy.
\end{itemize}

\subsection{Idea of this thesis}
This thesis is divided into two major blocks. As a first step we explain in Section \ref{sec:Auriga} how we model an analytic axisymmetric potential to a hydrodynamical cosmological simulation. At first thought, this seems to be trivial. But it became obvious that there are many struggles and problems to consider. Section \ref{sec:Dynamics} contains the main idea of this project: Testing if it is possible to constrain the gravitational potential of an external galaxy - where we treat simulated galaxies as external galaxies and apply the same techniques observers use - by making the accreted \ac{GC} distribution of one \ac{DG} a feature as sharp as possible in action space. This was tested and found to be working in \citet{Sanderson...gravpotstreams...2017} for streams. In Section \ref{sec:Discussion} we discuss our results address problems and give an outlook on how to continue this work.