\section{Introduction}\label{sec:Intro}
Galaxies are complex structures consisting of stars, gas, dust and \ac{DM} held together by gravity. The have many different shapes, colors and sizes, from low mass \ac{DG} to very massive elliptical galaxies, and are in constant change induced by stellar evolution but, with greater impact, by galaxy mergers. We can observe galaxies from our Galaxy, the \ac{MW}, over nearby galaxies, where we can still resolve individual parts, to high-redshift galaxies, when the Universe was still very young. This range of galaxies gives insight on galaxy formation and evolution through cosmic times. In their similarities, we can constrain many physical laws about galaxies and the Universe. 

\subsection{The importance of knowing the gravitational potential of galaxies}
The gravitational potential of galaxies (see Section \ref{subsubsec:pot_theory}) is fundamental to understand the structure of baryonic and invisible matter. It sets the foundation on how matter moves. We can observe the movement of stars and gas and draw conclusions on the total existing matter through the Poisson equation (Equation \ref{eq:Poisson} which connects the potential with density and matter. The potential influences many direct and indirect observables such as rotation curves and properties of the different components of a galaxy. Many empirical correlations for galaxies were found which rely on the mass (therefore potential) of a galaxy. 

\subsubsection{Dark matter}
Since until now we can measure \ac{DM} only indirect via its gravitational effect, it is important to measure the total mass and the potential of galaxies. We will now give a very quick overview on the discovery, most promising models, problems and alternatives. This Section closely follows the review chapter on \ac{DM} in Wilma Trick's PhD thesis \citep{Wilmathesis}; the main references are \citet{Ostriker...DM...2003, Maoz...astrophysics...2007} and \citet{Mo...galformev...2010} \\
\\\textbf{History of \ac{DM} discovery} In 1933, \citeauthor{Zwicky...DM...1933} observed the motions of galaxies in the Coma clusters and found a much higher velocity dispersion than what should be expected by visible matter after applying the virial theorem. He introduced the term "dunkle Materie" (German for dark matter) which described the invisible matter. Almost 40 years later, Rubin \textit{et al.} \citeyearpar{Rubin...DM...1970, Rubin...DM...1978, Rubin...DM...1980} measured rotation curves of first the Andromeda galaxy, our closest spiral galaxy, then of many other edge-on disk galaxies. The visible mass content would let the rotation curve decline towards higher radii $r$ but the rotation curves stayed constant over a large radial range following 
\begin{equation}\label{eq:circ_vel}
    v_{\mathrm{circ}}(r) = \sqrt{\frac{GM(r)}{r}} \sim \mathrm{constant}
\end{equation}
with the circular velocity $v_\mathrm{{circ}}(r)$, the gravitational constant $G$ and the total mass within the radius $M(r)$, indicating that there are spheroidal halos around galaxies build up from invisible matter. Other observational methods also rely on \ac{DM}, such as strong (e.g. \cite{Trick..stronglensing...2016}) and weak gravitational lensing \citep{Tyson...weaklensing...1990, Kaiser...weaklensing...1993}. \ac{DM} seems to only interact via gravitational forces but not with electromagnetic radiation and therefore cannot be observed by light. Unfortunately, up to now, there has not been a direct detection of \ac{DM} in any way which causes great challenges but also brings many opportunities of research.\\
\\\textbf{Cosmological aspects of \ac{DM}}
In the current standard model of cosmology, the \ac{LCDM}-model, the Universe is made up by dark energy ($\Lambda$) and matter. Recent measurements of the cosmic microwave background by the \citet{Planck...CMB...2018} found, that dark energy makes up the biggest amount of the energy density ($\Omega_\Lambda = 0.685$) and matter the rest ($\Omega_m = 0.315$), split up to $\Omega_b = 5\%$ baryonic matter and $\Omega_c = 26.5\%$ cold dark matter assuming a Hubble constant of H$_0$ =  \SI{67.27}{km.s^{-1}.Mpc^{-1}}. Therefore, \ac{DM} makes up around \SI{84}{\%} of the total matter in the Universe. \\
\\\textbf{Established \ac{DM} model - cold dark matter}
Cold dark matter (\acsu{CDM}) was first introduced by \cite{Davis....CDM...1985} through \textit{N}-body simulations. \ac{CDM} particles are long-lived and very massive (\SI{10}{GeV} to a few TeV). These particles decoupled very early in the beginning stages of the Universe, already before reionization, and therefore be nonrelativistic. Then, they clustered and merged and was built bottom-up in a hierarchical growth.  Relativistic particles would destroy small scale substructure which would lead to larger voids than we observe. Possible particle candidates are Weakly Interacting Massive Particles which are massive particles interacting only via the gravitational and the weak force. The predicted large scale structure predicted by \ac{CDM} simulations agrees extraordinary well with the observed clustering of galaxies. \textcolor{red}{mention NFW halo and that galaxies form in DM halos}\\
\\\textbf{Problems in the current model}
Even though the big success in explaining many phenomenons, \ac{CDM} has some problems on especially smaller scales (< \SI{1}{Mpc}) when comparing the predictions of cosmological \ac{DMO} simulations to observations (e.g., \citealp{Bullock...LCDMprobs...2017}). Some of these problems have been remedied in the recent years.
\begin{itemize}
    \item\textbf{The missing satellites problem:} These simulations predict many more satellites of disk galaxies in the low-mass end than we observe \citep{Klypin...missingsatellites...1999, Moore...missingsatellites..1999}. This can be explained by the fact that low mass \ac{DM} halos are extremely ineffective in forming galaxies and go completely dark below a certain threshold mass. In recent simulations analyzed by \citet{Sawala...noCDMproblems...2016} including baryons and physical prescriptions, the number of satellites matched the observations.
    \item \textbf{The cusp-core problem:} In these \ac{DMO} simulations, the halo density profile has a cusp in the center (citations) while observations find flatter density profiles and cored centers \citep{Flores...cuspcoreprob...1994, Moore...cuspcoreprob...1994}. A possible solution is including baryonic matter in these simulations which could be the driver for a cored center. 
    \item \textbf{The too-big-to-fail problem} \citep{Boylan...toobigtoofail...2011}\textbf{:} In the \ac{DMO} simulations, a large population of \acp{DM} satellites are found with greater central masses than any of the \ac{MW}'s dwarf spheroidals. These subhalos seem to have failed forming galaxies while halos with lower mass were successful. It was first found for the \ac{MW} but the same problem occurs for Andromeda \citep{Tollerud...M31tbtf...2014}, other Local Group galaxies \citep{Kirby...LGtbtf...2014} and in more isolated lower mass galaxies \citep{Ferrero...DGtbtf...2012, Papastergis...DGtbtf...2015, Papastergis...DGtbtf...2016}.
    \iffalse\item The planes of satellites problem: \fi
\end{itemize}

\textbf{\ac{CDM} alternatives} Many alternatives for \ac{CDM} were suggested and many of them are already ruled out. Some of the alternatives which still are considered are 
\begin{itemize}
    \item \textbf{Warm dark matter} These particles should have masses of around \SI{1}{keV}. The mass grows bottom up down to a characteristic mass scale, where below the free streaming of the particles prevents the halos to form and the \ac{DM} is distributed in a smooth background field instead \citep{Smith...WDM..2011, Schneider...WDM...2013}. This theory predicts less low mass \ac{DM} halos whose densities would be less cuspy in the centers due to higher thermal motions \citep{Bode...WDM...2001}.
    \item \textbf{\ac{MoND}:} \cite{Milgrom...MoND...1983} suggested the idea of a modified theory of Newtonian law of gravity which only has an effect in low accelerations. This theory explains flat rotation curves and how galaxies move in clusters. A big advantage would be the non necessity of a new mysterious dark particle. Nevertheless, there are phenomenons such as the Bullet cluster \citep{Clowe...Bullett...2006} which fit perfectly in the \ac{CDM} universe but have struggles with finding explanations in \ac{MoND}.
\end{itemize}

\subsubsection{Empirical galaxy correlations}
In galaxies, many characteristics are correlated. These correlations are usually found empirically by analyzing and combining observational results. Many of them include the mass of a galaxy so once we know the mass we also know about other properties of the galaxy. 
\begin{itemize}
    \item \textbf{\citet{Tully...Fisher...1977}} (TF) determined a relationship between the luminosity L of a spiral galaxy and its radial velocity (which is connected to the mass of a galaxy through Equation \ref{eq:circ_vel}):
    \begin{equation}
        L \propto (v_{\mathrm{circ, max}})^\beta \qquad \mathrm{with}\quad \beta =  2.5 - 5
    \end{equation}
    For the radial velocity, they measured the Doppler-broadened 21-cm radio emission line of neutral hydrogen (see Section \ref{subsec:mass_est_ext}). 
    \item \textbf{\citet{Faber...Jackson...1976}} measured the central radial velocity dispersion $\sigma_0$ of elliptical galaxies and found the relation to the luminosity  
    \begin{equation}
        L \propto \sigma_0^4,
    \end{equation}
    which is similar to the \acs{TF} relation of spiral galaxies. The derivation of this relation made simple assumptions such as a uniform mass distribution on the volume of radius \textit{R} and a constant mass-to-light ratio for all galaxies and equal surface brightnesses. Therefore, there is a large scatter in the data around this relation.
    \item \textbf{The Fundamental Plane} offers a better empirical fit to the data of ellipticals but needs another parameter, the effective radius $r_e$. It combines radius and luminosity of a galaxy with its gravitational well. Two representations of the fit are \citep{Carroll...Ostlie..2006}:
    \begin{align}
        L &\propto \sigma_0^{2.65}r_e^{0.65} \\
        r_e &\propto \sigma_0^{1.24}I_e^{-0.82}.
    \end{align}
    Dynamically hot stellar systems, i.e. stellar systems whose stars are on randomized orbits, follow this scaling relation \citep{Misgeld...hotss.FP...2011}.
    \iffalse\item M\_vir - N\_GC \fi
\end{itemize}

\iffalse
\subsubsection{Application}
MW \acp{GC} proper motions and dynamics (including action distribution and dynamical model of potentials): \cite{Vasiliev...GCdynsGaiaDR2...2018}\\
Modelling the \ac{MW}'s \ac{GC} system: \cite{Binney...GCsystem...2017}
\fi
\subsection{Dynamical modelling methods}
Since we are not able to measure every star in the \ac{MW} yet alone to resolve stars in external galaxies, we need to make models of the observations to extract information. The goal of these models is to measure the mass distribution based on observed kinematics. Stars in the disk and in the halo of galaxies can be considered as collisionless tracers which makes these models possible. \textcolor{red}{references}
\begin{itemize}
    \item \textbf{Jeans modelling} \citep{Jeans.....1915}: The first velocity moment of the \ac{CBE} (Equation \ref{eq:CBE}) relate the velocity ellipsoid of stars at a given position in a galaxy to the gravitational forces and the spatial \ac{DF} of the stars. One important advantage is that the computation is fast so a lot of different models can be explored. Disadvantage: The set of Jeans equations is not closed and therefore does not have a unique solution. Therefore, assumptions need to be made and the solution, if found, may give non-physical results for e.g. the \ac{DF}. 
    \item \textbf{Schwarzschild's orbital superposition approach} \citep{Schwarzschild...1979}: Dynamical models of triaxial galaxies can be made based on observed surface brightness distribution and observed kinematics. Given a potential, an orbit library over the full integral of motion space is constructed. The number/mass/light of stars on a specific orbit are described by a weight. These weighted orbits build up the stellar \ac{DF}. By comparing the surface brightness and kinematics of the model with the data the gravitational potential can be recovered. This method is mostly used in external galaxies \citep{Rix...Schwarzschild...1997, vdBosch...Schwarzschild...2008, Vasiliev...Schwarzschild...2013, Ling...Schwarzschild...2018}.
    \item \textbf{Action-based modelling:} Orbits in axisymmetric potentials can be modelled with \acp{DF} which arrange stars in 3D action space instead of 6D phase space \citep{Binney...actionbasedmodelling...2012, Bovy...actionbasedmodelling...2013}. The modelling is similar to the Schwarzschild approach. The differences are that it does not numerically integrate orbits but uses orbital actions and tori and instead of orbits weights, the analytic \acp{DF} are physically motivated and action-based. Since we need 6D phase space information to calculate actions it is mainly used in the \ac{MW} where we can resolve single stars. It is applied to modelling the disk (e.g., \citealp{trick...ROADMAPPING...2016, Wilmathesis}) but also to model the stellar halo (see Section \ref{subsec:mass_est_MW}). 
\end{itemize}
\subsection{Stellar objects in the halos of galaxies}\label{subsec:halo_objects}
\textbf{Globular clusters\acused{GC} (\acp{GC})} are self-gravitating, gas-free systems of $10^5$ to $10^7$ stars which are spherically grouped with a typical size of a few parsecs and mass around $10^5$ to $10^6\ \mathrm{M}_\odot$. Since they are some of the oldest stellar populations in the universe (approximately \SI{13}{Gyr} old), they contain much information about the assembly history and evolution of the \ac{MW} and external galaxies. \\\\
\textbf{Stellar streams} are remnants of tidally disrupted \acp{GC} or \acp{DG} and are a byproduct of hierarchical galaxy formation. A dynamically cold stream - which means it has a low intrinsic velocity dispersion - usually originates from a \ac{GC} \citep{Bonaca...streamsinfo...2018}. Their phase-space distribution is predominantly affected by the Galactic gravitational potential and depends less on internal kinematics \citep{Kupper...streams...2010, Kupper...streams...2012}. They are very thin and more than twice as long as wide so they can be treated as one dimensional in the plane of the sky \citep{Bonaca...streamsinfo...2018}. Hot stellar streams are created by satellites with higher velocity dispersions such as \acp{DG}. The first detected and since then often investigated (hot) stellar stream is the Sagittarius dwarf galaxy and its tidal arms \citep{Ibata...Sagittarius....1994}.

\subsection{Strategies to model the Milky Way potential}\label{subsec:mass_est_MW}
There are many different approaches to measure the mass and the potential of the Galaxy. Due to our position within the \ac{MW}, some methods which give very good constraints on overall parameters such as e.g. rotation curves of external galaxies (see Section \ref{subsec:mass_est_ext}) cannot be measured as easily. A big advantage is that we can resolve stellar positions and velocities with high precision, especially with \textit{Gaia} \citep{Gaia...mission...2016, GaiaDR2...overview...2018, GaiaDR...GCs...2018}, which is helpful in both Galactic archaeology and dynamical modelling. These are some of the kinematic and dynamical methods to measure the Galactic mass in the halo (their results are presented in Table \ref{tab:MW_mass_estimations}):
\begin{itemize}
    \iffalse\item Kinematics of nearby stars: \cite{Kuijken...LocalDMdens...1989, Bovy...LocalDMdens...2012} \fi
    \item \textbf{Orbits of stellar streams:} \citet{Johnston...MWstreams...1999} first found, that stellar streams contain information about the Galaxy's gravitational potential. In the case of kinematically cold streams, they move on orbits aligned with the remnants's orbit \citep{Eyre...streamstheo...2011}. It is therefore possible to get a rather direct measurement of the local acceleration close to the stream. So far, dynamical models of four single stellar streams have been used to constrain the mass and shape of the \ac{DM} halo (the Sagittarius dwarf galaxy \citep{Law...sagstream...2010, Gibbons...sagstream...2014, Dierickx...sagstream..2017}, the Orphan stream \citep{Newberg...orphanstream..2010}, the GD-1 stream \citep{Koposov...GD1stream...2010, Bowden...GD1stream...2015, Malhan...GD1stream...2018}, and the tails of the Palomar 5 globular cluster \citep{Kupper...pal5stream...2015}). Since these streams measure local properties, better constraints on the global potential can be achieved by looking at a population of streams \citep{Bonaca...streamsinfo...2018}.
    
    \item \textbf{Stellar streams in action space:} The pase-space distribution of tidal streams have the most simple form in action-angle-frequency space \citep{Tremaine...streamsactiontheory...1999, Helmi...streamsactionstheory...1999}. A deeper introduction to actions is given in Section \ref{sec:Dynamics}. Due to formerly high computing costs calculating actions numerically, this approach has been carried on larger scale out only recently after developing new, cheaper methods for action calculations (a review is given in \citealp{Sanders...actionreview...2016}). \citet{Streams...Sanders...2014} uses a St\"ackel-fitting algorithm \citep{Sanders...Staeckel...2012} to generate probabilistic models of streams to constrain the Galactic potential. \citet{Streams...Bovy...2014} introduces a new, general method of action-angle-frequency calculation for streams using an orbit-integration-based approximation. In \citet{Streams..GD1..Pal5...Bovy...2016}, this method is then for the first time applied individually and combined to the Palomar 5 and GD-1 streams.
    \item \textbf{\ac{GC} distribution:} Another tracer of the mass of the inner \ac{MW} halo (r $\le$ \SI{21}{kpc}) is the \ac{GC} distribution. With the second data release of \textit{Gaia} (DR2) distances and \acp{PM} are now available for 154 \acp{GC} \citep{Baumgardt...GCoverview...2019} respective 150 \acp{GC} \citep{Vasiliev...GCoverview...2018} which are nearly all known \ac{MW} \acp{GC}. 
    \citet{MWmass...GCmotions...Watkins...2018} use the kinematics of a subsample of \acp{GC} in this inner halo to constrain the Galaxy's mass. \citet{Posti...MWmassGCs...2018} employ another approach based on \citet{Binney...MWGCModel....2017} by fitting an action-based \ac{DF} to the 6D phase space data of 75 \acp{GC} to determine the mass and shape of the \ac{DM} halo. A very similar approach is carried out in \citet{Vasiliev...GCoverview...2018} which mainly differs in assumptions on the \ac{NFW} halo. \acp{PM} are measured by other telescopes as well. \citet{Sohn...GCsHST..2018} use Hubble Space Telescope \acp{PM} to derive the mass of the \ac{MW} with the same method as \citet{MWmass...GCmotions...Watkins...2018} but less \acp{GC}. 

    \item \textbf{Satellite dynamics:} To constrain the mass of the outer halo we can use the dynamics of satellite galaxies. The methods are similar to the ones used for \ac{GC} estimates. From position and velocity of the most distant dwarf Leo I (r $= 257.8_{{-35.1}}^{+16.8} $ kpc), \citet{GaiaDR...GCs...2018} provided a lower limit on the \ac{MW} mass. \citet{MWmass...sat...dyn} calculate in the hydrodynamical simulations \acp{DF} of specific energy and angular momentum, with given 6D phase space information, which vary according to the galaxy host mass, estimate this mass by a maximum likelihood and apply this method to the \ac{MW}.
\end{itemize}

An overview of the results is shown in Table \ref{tab:MW_mass_estimations}.
\begin{table}[htbp]
\captionsetup{format=plain}
    \centering
    \begin{tabular}{@{}llll@{}}
         \toprule
         Method& Virial mass [$10^{12} \mathrm{M}_\odot$] & Rotational velocity [km s$^{-1}$]&Reference  \\
         
         \midrule

         \bottomrule 
    \end{tabular}
    \caption{Mass estimation results of the \ac{MW} }
    \label{tab:MW_mass_estimations}
\end{table}


\subsection{Strategies to measure the mass of external galaxies}\label{subsec:mass_est_ext}
In external galaxies, we cannot resolve single stars but, depending on mass, luminosity and distance of the galaxy, we can go from observing the whole galaxy as a point source in the sky down to resolving objects in the halo of a galaxy such as \acp{GC}. To get a constraint on their mass it is useful to measure the rotational velocity of the galaxy (see Equation \ref{eq:circ_vel}). Different techniques evolved over time and telescope resolution, from measuring one value of the velocity to having spectra for each observed pixel.
\begin{itemize}
    \item\textbf{1D: 21cm line} Hydrogen is the simplest yet most abundant atom in space. The 21 cm line of neutral hydrogen (HI line) is visible through photons which are emitted when relative spins change from parallel to antiparallel. HI is detectable in radio bands in most spirals and some ellipticals. Line-of-sight velocities can be measured from the Doppler shift of these emission lines which give us a constraint on the galaxy's disk maximum rotation velocity and therefore a measurement of the enclosed mass.
    \item\textbf{2D: slit along the major axis} With the slit of a spectrograph aligned along the major axis of a galaxy, it is possible to take stellar spectra at different galactocentric radii. These spectra give us for a population of stars the line-of-sight velocity and the velocity dispersion which both are useful in dynamical modelling. Stars moving towards the observer are blue-shifted, stars moving away are redshifted. From the Doppler shift one can derive the rotational velocity and therefore the mass. 
    \item \textbf{3D: \acl{IFU}} \ac{IFU} spectrographs observe the 2D field of view and take a spectrum for each pixel at the same time. With that method we gain a 2D velocity map and can learn more about the mass distribution of the observed galaxy.
\end{itemize}

\subsection{Idea of this thesis: Adaptive dynamics of accreted globular clusters}
\textbf{Context}
As we have seen, there are several methods of measuring the mass of external galaxies which rely mainly on the rotational velocities. We can try to adapt some of the methods which we use in the \ac{MW} to external galaxies. These would be ideas which rely on objects in the (outer) halos of galaxies, which still can be resolved, \acp{GC}. 
\\
High resolution \ac{IFU} data (e.g. from the Multi Unit Spectroscopic Explorer (MUSE, \citealp{Bacon...MUSE...2010}) of external galaxies provides us with wealthy information suhc as high resolution 2D positions and radial velocities of \acp{GC} (e.g. in the Fornax galaxies - see Figure 15 of \citealp{Sarzi...Fornax3d....2018}). Despite in the very crowded inner regions, probably all \acp{GC} can be observed so there are no problems with completeness and selection effects. Nevertheless, there is no 6D phase-space data available yet. 
\\
\citet{Leaman...agemetall.MWGCs...2013} use the age-metallicity relation of the \ac{MW} \acp{GC} to distinguish them between in-situ and ex-situ formation and to relate them to their progenitor. This method can be used in external galaxies to find \acp{GC} accreted from the same satellite.
\\
After the merger, \acp{GC} in the \ac{MW} but also in external galaxies retain a dynamical memory of their progenitor. Their phase-space distribution on the halo depends on the mass and shape of the host. As in Section \ref{subsec:mass_est_MW}, we can use these \acp{GC} to constrain the gravitational potential.
\\\\
\textbf{Idea}
In this work, we test the idea of adaptive dynamics \citep{Binney...adaptivedynamics...2005} in external galaxies. Adaptive dynamics uses dynamical features (action-angle-frequencies) of accreted objects in the halo of galaxies (streams and \acp{GC}) and by making these features as sharp as possible and minimize their spread we should be able to constrain the gravitational potential of the galaxy. This implies, that the \ac{DF} of these accreted objects is a $\delta$-function. If this method works, we will gain more insight in the merger history of the galaxies and on the \ac{DM} mass and distribution. We test this method in a hydrodynamical, cosmological simulation, where we have full 6D phase-space information. To make our investigations comparable to the methods observers use in external galaxies, we fit an analytic axisymmetric potential to the simulation.
\\\\
\textbf{Structure}
This thesis is divided into two major blocks. As a first step we explain in Section \ref{sec:Auriga} how we model an analytic axisymmetric potential to a hydrodynamical. cosmological simulation. At first thought, this seems to be trivial. But it became obvious that there are many struggles and problems to consider. Section \ref{sec:Dynamics} contains the main idea of this project: Testing if it is possible to constrain the gravitational potential of an external galaxy - where we treat simulated galaxies as external galaxies and apply the same techniques observers use - by making the accreted \ac{GC} distribution of one \ac{DG} a feature as sharp as possible in action space. In Section \ref{sec:Discussion} we discuss our results address problems and give an outlook on how to continue this work. A short summary and conclusion is given in Section \ref{sec:sumconc}.