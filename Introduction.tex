\section{Introduction}\label{sec:Intro}
Galaxies are complex structures consisting of stars, gas, dust and \ac{DM} held together through gravity. The have many different shapes, sizes and colors and are in constant change. We can observe galaxies from our Galaxy, the \ac{MW}, over nearby galaxies, where we can still resolve individual parts, to high-redshift galaxies, when the Universe was still very young. This range of galaxies gives insight on galaxy formation and evolution through cosmic times. In their similarities, we can constrain many physical laws about galaxies and the Universe. 

\subsection{Why do we want to have a potential}
The gravitational potential of galaxies is fundamental to understand the structure of baryonic and invisible matter. It sets the foundation on how matter moves. We can observe the movement of stars and gas and draw conclusions on the total existing matter. The potential influences many direct and indirect observables such as rotation curves and properties of the different components of a galaxy. Many empirical correlations for galaxies were found which rely on the mass (therefore potential) of a galaxy. 

\subsubsection{Mass estimates of dark matter}
Since until now we can measure \ac{DM} only indirect via its gravitational effet, it is important to measure the total mass and the potential of galaxies. \ac{DM} is a challenging but fascinating idea. We will now give a very quick overview on the discovery, most promising models, problems and alternatives. \\
\\\textbf{History of \ac{DM} discovery} In 1933, \citeauthor{Zwicky...DM...1933} observed the motions of galaxies in the Coma clusters and found a much higher velocity dispersion than what should be expected by visible matter after applying the virial theorem. He introduced the term "dunkle Materie" (German for dark matter) which described the invisible matter. Almost 40 years later, Rubin \textit{et al.} \citeyearpar{Rubin...DM...1970, Rubin...DM...1978, Rubin...DM...1980} measured rotation curves of first the Andromeda galaxy, our closest spiral galaxy, then of many other edge-on disk galaxies. The visible mass content would let the rotation curve decline towards higher radii but the rotation curves stayed constant over a large range. The best and nowadays established explanation for this behaviour is the presence of \ac{DM}. Other observational methods also rely on \ac{DM}, such as strong (e.g. \citep{Trick..stronglensing...2016}) and weak gravitational lensing \citep{Tyson...weaklensing...1990, Kaiser...weaklensing...1993}. \ac{DM} seems to only interact via gravitational forces but not with electromagnetic radiation and therefore cannot be observed by light. Unfortunately, up to now, there has not been a direct detection of \ac{DM} in any way which causes great challenges but also brings many opportunities of research.\\
\\\textbf{Cosmological aspects of \ac{DM}}
In the current standard model of cosmology, the \ac{LCDM}-model, the Universe is made up by dark energy ($\Lambda$) and matter. Recent measurements of the \ac{CMB} by the \citet{Planck...CMB...2018} found, that dark energy makes up the biggest amount of the energy density ($\Omega_\Lambda = 0.685$) and matter the rest ($\Omega_m = 0.315$), split up to $\Omega_b = 5\%$ baryonic matter and $\Omega_c = 26.5\%$ cold dark matter assuming a Hubble constant of H$_0$ =  \SI{67.27}{km.s^{-1}.Mpc^{-1}}. Therefore, \ac{DM} makes up around \SI{84}{\%} of the total matter in the Universe. \\
\\\textbf{Established \ac{DM} model - cold dark matter}
\ac{CDM} was first introduced by \cite{Davis....CDM...1985} through \textit{N}-body simulations. \ac{CDM} particles are long-lived and very massive (\SI{10}{GeV} to a few TeV). These particles decoupled very early in the beginning stages of the Universe, already before reionization, and therefore be nonrelativistic. Then, they clustered and merged and was built bottom-up in a hierarchical growth.  Relativistic particles would destroy small scale substructure which would lead to larger voids than we observe. Possible particle candidates are \ac{WIMPs} which are massive particles interacting only via the gravitational and the weak force. The predicted large scale structure predicted by \ac{CDM} simulations agrees extraordinary well with the observed clustering of galaxies. 

\\\textbf{Problems in the current model}
Even though the big success in explaining many phenomenons, \ac{CDM} has some problems on especially smaller scales (< \SI{1}{Mpc}) when comparing the predictions of cosmological \ac{DMO} simulations to observations ( e.g., \cite{Bullock...LCDMprobs...2017}). 
\begin{itemize}
    \item The missing satellites problem: these simulations predict many more satellites of disk galaxies in the low-mass end than we observe \citep{Klypin...missingsatellites...1999, Moore...missingsatellites..1999}. This can be explained by the fact that low mass \ac{DM} halos are extremely insufficient in forming galaxies and go completely dark below a certain threshold mass. In recent simulations analyzed by \citet{Sawala...noCDMproblems...2016} including baryons and physical prescriptions, the number of satellites matched the observations.
    \item The cusp-core problem: in these simulations, the halo density profile has a cusp in the center (citations) while observations find flatter density profiles and cored centers \citep{Flores...cuspcoreprob...1994, Moore...cuspcoreprob...1994}. A possible solution is including baryonic matter in these simulations which could be the driver for a cored center. 
    \item The too-big-to-fail problem \citep{Boylan...toobigtoofail...2011}: In the \ac{DMO} simulations, a large population of \acp{DM} satellites are found with greater central masses than any of the \ac{MW}'s dwarf spheroidals. These subhalos seem to have failed forming galaxies while halos with lower mass were successful. It was first found for the \ac{MW} but the same problem occurs for Andromeda \citep{Tollerud...M31tbtf...2014}, other \ac{LG} galaxies \citep{Kirby...LGtbtf...2014} and in more isolated lower mass galaxies \citep{Ferrero...DGtbtf...2012, Papastergis...DGtbtf...2015, Papastergis...DGtbtf...2016}.
    \iffalse\item The planes of satellites problem: \fi
\end{itemize}

\textbf{\ac{CDM} alternatives} Many alternatives for \ac{CDM} were suggested and many of them are already ruled out. Some of the alternatives which still are considered are 
\begin{itemize}
    \item \ac{WDM}: These particles should have masses of around \SI{1}{keV}. The mass grows bottom up down to a characteristic mass scale, where below the free streaming of the particles prevents the halos to form and the \ac{DM} is distributed in a smooth background field instead \citep{Smith...WDM..2011, Schneider...WDM...2013}. This theory predicts less low mass \ac{DM} halos whose densities would be less cuspy in the centers due to higher thermal motions \citep{Bode...WDM...2001}.
    \item \ac{MoND}: \cite{Milgrom...MoND...1983} suggested the idea of a modified theory of Newtonian law of gravity which only has an effect in low accelerations. This theory explains flat rotation curves and how galaxies move in clusters. A big advantage would be the non necessity of a new mysterious dark particle. Nevertheless, there are phenomenons such as the Bullet cluster \citep{Clowe...Bullett...2006} which fit perfectly in the \ac{CDM} universe but have struggles with finding explanations in \ac{MoND}.
\end{itemize}

\subsubsection{Empirical correlations}
In galaxies, many characteristics are correlated. There correlations are usually found empirically by analyzing and combining observational results
\begin{itemize}
    \item Fundamental plane
    \item Tully-Fisher
    \item Faber-Jackson
\end{itemize}

\iffalse
\subsubsection{Application}
MW \acp{GC} proper motions and dynamics (including action distribution and dynamical model of potentials): \cite{Vasiliev...GCdynsGaiaDR2...2018}\\
Modelling the \ac{MW}'s \ac{GC} system: \cite{Binney...GCsystem...2017}
\fi

\subsection{Milky Way mass estimates}\label{subsec:mass_est_MW}
There are many different approaches to measure the mass and the potential of the Galaxy. Due to our position within the \ac{MW}, some methods which give very good constraints on overall parameters such as e.g., rotation curves, of external galaxies (see Section \ref{subsec:mass_est_ext}) cannot be measured as easily. A big advantage is that we can resolve stellar positions and velocities with high precision which is helpful in both Galactic archaeology and dynamical modelling. These are some of the methods to measure the Galactic mass:
\begin{itemize}
    \item Kinematics of nearby stars: \cite{Kuijken...LocalDMdens...1989, Bovy...LocalDMdens...2012}
    \item Stellar streams: Stellar streams are remnants of disrupted \acp{GC}. 
    \begin{itemize}
        \item Sanders \citep{Streams...Sanders...2014}
        \item Bovy \citep{Streams...Bovy...2014}
        \item Bovy+ \citep{Streams..GD1..Pal5...Bovy...2016}
    \end{itemize}
    \item \ac{GC} distribution 
    \begin{itemize}
        \item Mass \& shape of \ac{MW} \ac{DM} halo with \acp{GC} \textit{Gaia} + Hubble: \cite{Posti...MWmassGCs...2018}
        \item GC motions \citep{MWmass...GCmotions...Watkins...2018}
    \end{itemize}
    \item satellite dynamics \citep{MWmass...sat...dyn}
    \item action-based modelling
    \begin{itemize}
        \item Disk: Trick \citep{Wilmathesis}
        \item DM halo \citep{Sanderson...gravpotstreams...2017} 
    \end{itemize}
    
\end{itemize}


   
Bonaca: cold stream kann immer noch als single orbit modelliert werden 
in zwer galaxien nicht moeglich, gibt es cold GC stream dwarf galaxy mergers


\subsection{Mass estimated of external galaxies}\label{subsec:mass_est_ext}
\subsubsection{How to measure velocities of external galaxies}
\textbf{1D: 21cm line}\\
\textbf{2D: slit along the major axis}\\
\textbf{3D: \ac{IFU}}

\subsection{Dynamical modelling methods}
\subsubsection{Jeans}
\subsubsection{Schwarzschild}
\subsubsection{Action-based modelling}

\subsection{Idea of this thesis}
This thesis contains two major blocks. As a first step we explain how we model an analytic axisymmetric potential to a hydrodynamical cosmological simulation. 