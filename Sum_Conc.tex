\section{Summary and conclusion}\label{sec:sumconc}
We tested adaptive dynamics in external galaxies to see if it enables us to constrain the potential of these galaxies with the actions of accreted \acp{GC}. We did these investigations in a simulation of the cosmological magneto-hydrodynamic Auriga simulation suite. We first fitted an analytical, axisymmetric, three component potential to the simulation to have a galaxy similar to how observers model external galaxies. In this model, we carried out several test on accreted \acp{GC} in action space. We looked at their action distribution, tried to minimize their spread in action space, investigated the time evolution and picked a small subgroup of \acp{GC} which are one similar orbits at the current time to see how they involved. We carried out several tests, such as energy evolution and analyzing actions in a fixed potential, to make sure these problems do not come from some certain assumptions that we took. The main conclusions are that in this simulation,
\begin{itemize}
    \item fitting an analytic axisymmetric potential to a \ac{MW}-like simulation is not trivial and requires a good techniques but also some compromises;
    \item accreted \acp{GC} from one progenitor do not move on same orbit. Their \ac{DF} must be more complex;
    \item the orbits of these accreted \acp{GC} do not stay constant.
\end{itemize}
Still, further investigation is needed to make this result unambiguous.