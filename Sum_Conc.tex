\section{Summary and conclusion}\label{sec:sumconc}
We tested adaptive dynamics in external galaxies to see if it enables us to constrain the potential of these galaxies with the actions of accreted \acp{GC}. The idea of adaptive dynamics is to use these accreted particles from the same progenitor to constrain the gravitational potential of the host galaxy. It assumes that the \acp{DF} of those objects are $\delta$-functions in action space. Making their action distributions as small as possible should constrain the potential parameters (e.g. mass and shape, depending on the chosen analytic potential). 
\\We did these investigations in a galaxy simulation of the cosmological magneto-hydrodynamic Auriga simulation suite. We first fitted an analytic axisymmetric, three component potential model to the simulation motivated by the simplified assumptions dynamical modellers and observers often make to describe external galaxies or the \ac{MW}. In this potential model, we carried out several test on accreted \acp{GC} from \ac{DG} progenitors in action space. We looked at their action distribution, tried to minimize their spread in action space, investigated the time evolution and picked a small subgroup of \acp{GC} which are one similar orbits at the current time to see how they evolved. We carried out several tests, such as energy evolution and analyzing actions in a fixed potential, to make sure our results are not affected by certain assumptions like the \ac{GC} selection and our potential model. The main conclusions are that,
\begin{itemize}
    \item fitting an analytic axisymmetric potential to a \ac{MW}-like simulation is not trivial and requires good techniques but also some compromises. We propose a simple and physically motivated strategy in this work;
    \item accreted \acp{GC} from one progenitor do not move on the same orbit. Their \ac{DF} is expected to be more complex which would need to be accounted for in dynamical modelling attempts in the future;
    \item the orbits of these accreted \acp{GC} do not stay constant suggesting that various and more complex physical processes are at play in shaping the \ac{GC} distribution.
\end{itemize}
