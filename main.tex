\documentclass[a4paper,12pt,abstracton]{scrartcl}
\usepackage[ngerman, english]{babel}
\usepackage[utf8]{inputenc}
\usepackage[T1]{fontenc}
\usepackage{lipsum}
\usepackage{blindtext}
\usepackage{graphicx}
\usepackage{color}
\usepackage{setspace}
\usepackage{hyperref}
\usepackage[printonlyused]{acronym}
\usepackage{amsmath}
\usepackage{amsfonts}
\usepackage{amssymb}
\usepackage[export]{adjustbox}
\usepackage{subcaption}
\usepackage{makecell}
\usepackage[]{units}
\usepackage{natbib}
\usepackage{aas_macros}
\usepackage{enumerate}
\usepackage{verbatim}

\title{MSc thesis}
\author{Sophia Milanov}
\date{\today}

\begin{document}

\bibliographystyle{glenn} 
\pagenumbering{roman}

\onehalfspacing
%% this will generate title pages similar to the template provided
%% by the Department of Physics and Astronomy Heidelberg
%%
%% More information:
%% http://www.physik.uni-heidelberg.de/aktuelles/studium/
%% (PDF link: ...studium/download/145/Vorlage_Diplomarbeit_Formular.pdf)

%% Titleintro
\begin{titlepage}
\thispagestyle{empty}
\begin{center}
  \renewcommand{\baselinestretch}{2.00}
  \Large\sffamily
  Department of Physics and Astronomy\\
  \large University of Heidelberg
  \par\vfill\normalfont
  Master thesis in Physics\\
  submitted by\\
  \textbf{Sophia Lilleengen}\\
  born in D\"usseldorf (Germany)\\
  \textbf{2019}
\end{center}
\newpage
\thispagestyle{empty}
\mbox{}
\newpage
%% Titlepage
\thispagestyle{empty}
\begin{center}
  \renewcommand{\baselinestretch}{2.00}
  \Large\bfseries\sffamily
    Consequences\\
    of\\
    my actions
  \par
  \vfill
  \large\normalfont
  This Master thesis has been carried out by Sophia Lilleengen at the\\
  European Southern Observatory\\
  under the supervision of\\
  Dr. Glenn van de Ven and Dr. Wilma Trick
  %% additionally insert second supervisor here if carrying out an
  %% external diploma thesis. Reduce vspace in L. 44 accordingly.
\end{center}\par
\vspace{5\baselineskip}

% reset baselinestretch
\renewcommand{\baselinestretch}{1.00}\normalsize
\end{titlepage}

\begin{abstract}
\hspace{-12pt}Many astrophysical and galaxy-scale cosmological problems require a well determined gravitational potential. Globular clusters (GCs) surrounding galaxies can be used as dynamical tracers of the total and dark matter distribution at large (kpc) scales. This M.Sc. project investigates - by means of the Auriga galaxy simulations and in anticipation of high-resolution IFU data of external galaxies - if a novel action-based approach could tightly constrain the gravitational potential. In the Milky Way, action-angles of resolved stars in stellar streams were found to be useful in constraining its gravitational potential. In external galaxies, no individual stars but only GCs can be resolved (2D projected position and line-of-sight velocities as observables). Actions are integrals of motion with an intuitive physical meaning - e.g. the radial action quantifies the radial oscillation of the orbit - and therefore make excellent orbit labels. In a slowly varying potential, actions are assumed to stay constant. We find, that within the Auriga simulations actions show a significant variation over time and we cannot rely on them being constant. Moreover, varying the gravitational potential of the host within reasonable ranges does not significantly change the distribution of GCs in action space. Testing this approach also in Gaia DR2 by calculating the actions of Gaia-Enceladus in different MW potentials we confirm that we cannot distinguish between them in GC action space distribution. We find, that we cannot use this method to constrain the gravitational potential of galaxies. In reverse, calculating actions does not require a well-determined gravitational potential so we can use them as strong probes for other astrophysical problems. 

%\vspace{2cm}
\begin{center}
 \textbf{Zusammenfassung}\end{center}

\hspace{-12pt}\blindtext

\end{abstract}

\newpage
\addtocontents{toc}{}
\tableofcontents

\newpage
\pagenumbering{arabic}

\section{Introduction}\label{sec:Intro}
Galaxies are complex structures consisting of stars, gas, dust and \ac{DM} held together through gravity. The have many different shapes, sizes and colors and are in constant change. We can observe galaxies from our Galaxy, the \ac{MW}, over nearby galaxies, where we can still resolve individual parts, to high-redshift galaxies, when the Universe was still very young. This range of galaxies gives insight on galaxy formation and evolution through cosmic times. In their similarities, we can constrain many physical laws about galaxies and the Universe. 

\subsection{Why do we want to have a potential}
The gravitational potential of galaxies is fundamental to understand the structure of baryonic and invisible matter. It sets the foundation on how matter moves. We can observe the movement of stars and gas and draw conclusions on the total existing matter. The potential influences many direct and indirect observables such as rotation curves and properties of the different components of a galaxy. Many empirical correlations for galaxies were found which rely on the mass (therefore potential) of a galaxy. 

\subsubsection{Mass estimates of dark matter}
Since until now we can measure \ac{DM} only indirect via its gravitational effet, it is important to measure the total mass and the potential of galaxies. \ac{DM} is a challenging but fascinating idea. We will now give a very quick overview on the discovery, most promising models, problems and alternatives. \\
\\\textbf{History of \ac{DM} discovery} In 1933, \citeauthor{Zwicky...DM...1933} observed the motions of galaxies in the Coma clusters and found a much higher velocity dispersion than what should be expected by visible matter after applying the virial theorem. He introduced the term "dunkle Materie" (German for dark matter) which described the invisible matter. Almost 40 years later, Rubin \textit{et al.} \citeyearpar{Rubin...DM...1970, Rubin...DM...1978, Rubin...DM...1980} measured rotation curves of first the Andromeda galaxy, our closest spiral galaxy, then of many other edge-on disk galaxies. The visible mass content would let the rotation curve decline towards higher radii but the rotation curves stayed constant over a large range. The best and nowadays established explanation for this behaviour is the presence of \ac{DM}. Other observational methods also rely on \ac{DM}, such as strong (e.g. \citep{Trick..stronglensing...2016}) and weak gravitational lensing \citep{Tyson...weaklensing...1990, Kaiser...weaklensing...1993}. \ac{DM} seems to only interact via gravitational forces but not with electromagnetic radiation and therefore cannot be observed by light. Unfortunately, up to now, there has not been a direct detection of \ac{DM} in any way which causes great challenges but also brings many opportunities of research.\\
\\\textbf{Cosmological aspects of \ac{DM}}
In the current standard model of cosmology, the \ac{LCDM}-model, the Universe is made up by dark energy ($\Lambda$) and matter. Recent measurements of the \ac{CMB} by the \citet{Planck...CMB...2018} found, that dark energy makes up the biggest amount of the energy density ($\Omega_\Lambda = 0.685$) and matter the rest ($\Omega_m = 0.315$), split up to $\Omega_b = 5\%$ baryonic matter and $\Omega_c = 26.5\%$ cold dark matter assuming a Hubble constant of H$_0$ =  \SI{67.27}{km.s^{-1}.Mpc^{-1}}. Therefore, \ac{DM} makes up around \SI{84}{\%} of the total matter in the Universe. \\
\\\textbf{Established \ac{DM} model - cold dark matter}
\ac{CDM} was first introduced by \cite{Davis....CDM...1985} through \textit{N}-body simulations. \ac{CDM} particles are long-lived and very massive (\SI{10}{GeV} to a few TeV). These particles decoupled very early in the beginning stages of the Universe, already before reionization, and therefore be nonrelativistic. Then, they clustered and merged and was built bottom-up in a hierarchical growth.  Relativistic particles would destroy small scale substructure which would lead to larger voids than we observe. Possible particle candidates are \ac{WIMPs} which are massive particles interacting only via the gravitational and the weak force. The predicted large scale structure predicted by \ac{CDM} simulations agrees extraordinary well with the observed clustering of galaxies. 

\\\textbf{Problems in the current model}
Even though the big success in explaining many phenomenons, \ac{CDM} has some problems on especially smaller scales (< \SI{1}{Mpc}) when comparing the predictions of cosmological \ac{DMO} simulations to observations ( e.g., \cite{Bullock...LCDMprobs...2017}). 
\begin{itemize}
    \item The missing satellites problem: these simulations predict many more satellites of disk galaxies in the low-mass end than we observe \citep{Klypin...missingsatellites...1999, Moore...missingsatellites..1999}. This can be explained by the fact that low mass \ac{DM} halos are extremely insufficient in forming galaxies and go completely dark below a certain threshold mass. In recent simulations analyzed by \citet{Sawala...noCDMproblems...2016} including baryons and physical prescriptions, the number of satellites matched the observations.
    \item The cusp-core problem: in these simulations, the halo density profile has a cusp in the center (citations) while observations find flatter density profiles and cored centers \citep{Flores...cuspcoreprob...1994, Moore...cuspcoreprob...1994}. A possible solution is including baryonic matter in these simulations which could be the driver for a cored center. 
    \item The too-big-to-fail problem \citep{Boylan...toobigtoofail...2011}: In the \ac{DMO} simulations, a large population of \acp{DM} satellites are found with greater central masses than any of the \ac{MW}'s dwarf spheroidals. These subhalos seem to have failed forming galaxies while halos with lower mass were successful. It was first found for the \ac{MW} but the same problem occurs for Andromeda \citep{Tollerud...M31tbtf...2014}, other \ac{LG} galaxies \citep{Kirby...LGtbtf...2014} and in more isolated lower mass galaxies \citep{Ferrero...DGtbtf...2012, Papastergis...DGtbtf...2015, Papastergis...DGtbtf...2016}.
    \iffalse\item The planes of satellites problem: \fi
\end{itemize}

\textbf{\ac{CDM} alternatives} Many alternatives for \ac{CDM} were suggested and many of them are already ruled out. Some of the alternatives which still are considered are 
\begin{itemize}
    \item \ac{WDM}: These particles should have masses of around \SI{1}{keV}. The mass grows bottom up down to a characteristic mass scale, where below the free streaming of the particles prevents the halos to form and the \ac{DM} is distributed in a smooth background field instead \citep{Smith...WDM..2011, Schneider...WDM...2013}. This theory predicts less low mass \ac{DM} halos whose densities would be less cuspy in the centers due to higher thermal motions \citep{Bode...WDM...2001}.
    \item \ac{MoND}: \cite{Milgrom...MoND...1983} suggested the idea of a modified theory of Newtonian law of gravity which only has an effect in low accelerations. This theory explains flat rotation curves and how galaxies move in clusters. A big advantage would be the non necessity of a new mysterious dark particle. Nevertheless, there are phenomenons such as the Bullet cluster \citep{Clowe...Bullett...2006} which fit perfectly in the \ac{CDM} universe but have struggles with finding explanations in \ac{MoND}.
\end{itemize}

\subsubsection{Empirical correlations}
In galaxies, many characteristics are correlated. There correlations are usually found empirically by analyzing and combining observational results
\begin{itemize}
    \item Fundamental plane
    \item Tully-Fisher
    \item Faber-Jackson
\end{itemize}

\iffalse
\subsubsection{Application}
MW \acp{GC} proper motions and dynamics (including action distribution and dynamical model of potentials): \cite{Vasiliev...GCdynsGaiaDR2...2018}\\
Modelling the \ac{MW}'s \ac{GC} system: \cite{Binney...GCsystem...2017}
\fi

\subsection{Milky Way mass estimates}\label{subsec:mass_est_MW}
There are many different approaches to measure the mass and the potential of the Galaxy. Due to our position within the \ac{MW}, some methods which give very good constraints on overall parameters such as e.g., rotation curves, of external galaxies (see Section \ref{subsec:mass_est_ext}) cannot be measured as easily. A big advantage is that we can resolve stellar positions and velocities with high precision which is helpful in both Galactic archaeology and dynamical modelling. These are some of the methods to measure the Galactic mass:
\begin{itemize}
    \item Kinematics of nearby stars: \cite{Kuijken...LocalDMdens...1989, Bovy...LocalDMdens...2012}
    \item Stellar streams: Stellar streams are remnants of disrupted \acp{GC}. 
    \begin{itemize}
        \item Sanders \citep{Streams...Sanders...2014}
        \item Bovy \citep{Streams...Bovy...2014}
        \item Bovy+ \citep{Streams..GD1..Pal5...Bovy...2016}
    \end{itemize}
    \item \ac{GC} distribution 
    \begin{itemize}
        \item Mass \& shape of \ac{MW} \ac{DM} halo with \acp{GC} \textit{Gaia} + Hubble: \cite{Posti...MWmassGCs...2018}
        \item GC motions \citep{MWmass...GCmotions...Watkins...2018}
    \end{itemize}
    \item satellite dynamics \citep{MWmass...sat...dyn}
    \item action-based modelling
    \begin{itemize}
        \item Disk: Trick \citep{Wilmathesis}
        \item DM halo \citep{Sanderson...gravpotstreams...2017} 
    \end{itemize}
    
\end{itemize}


   
Bonaca: cold stream kann immer noch als single orbit modelliert werden 
in zwer galaxien nicht moeglich, gibt es cold GC stream dwarf galaxy mergers


\subsection{Mass estimated of external galaxies}\label{subsec:mass_est_ext}
\subsubsection{How to measure velocities of external galaxies}
\textbf{1D: 21cm line}\\
\textbf{2D: slit along the major axis}\\
\textbf{3D: \ac{IFU}}

\subsection{Dynamical modelling methods}
\subsubsection{Jeans}
\subsubsection{Schwarzschild}
\subsubsection{Action-based modelling}

\subsection{Idea of this thesis}
This thesis contains two major blocks. As a first step we explain how we model an analytic axisymmetric potential to a hydrodynamical cosmological simulation. 

%\section{Theory}
%\subsection{Galaxies}
%\subsection{Halo build up}
%\subsection{Gravitational potential of galaxies}


\section{An axisymmetric potential for a cosmological simulation.}
Code from Timo Halbesma, Federico Marinacci, Rob Grand, Wilma Trick


\subsection{About Auriga}\label{subsec:auria}
\subsubsection{Hydrodynamical galaxy simulations}\label{subsubsec:hydro_sim}
To understand how our Universe and everything in it has formed and evolved, astronomers use simulations of it two ways: trying to match observations of real galaxies and thus checking if the input 'recipes' are correct and predicting observations which then are to be found by observers. These simulations stretch over a large range of astronomical scales, from stars and planets to the evolution of the cosmic web, but also over different numerical techniques, from more empirical, statistical Monte-Carlo methods to cosmological hydrodynamical \textit{N}-body simulations. To learn more about the formation and evolution of galaxies, these hydrodynamical cosmological simulations are a wealthy tool to exploit. 

\\This work focuses on the Auriga \citep{AurigaGrand} simulations, which try to recreate spiral galaxies such as our own. 

\subsubsection{Auriga}\label{subsubsec:auriga_intro}
Auriga \citep{AurigaGrand} is a magnetohydrodynamical zoom-in simulation of an isolated \ac{MW} like galaxy. It is build with AREPO \citep{AREPO} and includes galaxy physics, \ac{AGN} feedback and magnetic fields. Its goal is to match the observables of the \ac{MW} today and to produce its history which can be compared to observations of spiral galaxies in earlier stages of development. The snapshots go from redshift 127, which is close to the beginning of the universe, to redshift 0, today. All 30 galaxies are run in normal resolution and 3 selected are run in low and high resolution as well. At redshift = 0, different galaxy shapes are evolved. Most of them are spirals but a few are in a merger process. All galaxies have a rich merger history.


\begin{figure}
    \centering
    \begin{subfigure}[b]{0.8\textwidth}
	    \includegraphics[width=\textwidth]{plots/Auriga/DM_and_stars_xy_distribution.png}
	    \label{fig:DM_stars_xy}
    \end{subfigure}
    
    \begin{subfigure}[b]{0.8\textwidth}
    \centering
    	\includegraphics[width=\textwidth]{plots/Auriga/DM_and_stars_xz_distribution.png}
    	\label{fig:DM_stars_xy}
    \end{subfigure}
    \caption{\ac{DM} (grey) and stellar (colors) particle distribution of the whole simulation Auriga24 at $\textit{z}=0$. The \ac{DM} forms the cosmic web, where the mass gathers along its filaments. Baryonic matter also follows these structures. At the most massive parts of the \ac{DM} distribution, the most stellar particles fell in. }\label{fig:DM_stars_AU24}
\end{figure}


\begin{figure}
    \centering
    \begin{subfigure}[b]{0.8\textwidth}
	    \includegraphics[width=\textwidth]{plots/Auriga/Au24_stars_xy_distribution_halo0.png}
	    \label{fig:Au24_stars_xy}
    \end{subfigure}
    
    \begin{subfigure}[b]{0.8\textwidth}
    \centering
    	\includegraphics[width=\textwidth]{plots/Auriga/Au24_stars_xz_distribution_halo0.png}
    	\label{fig:Au24_stars_xz}
    \end{subfigure}
    \caption{Stellar distribution of 0th halo at $\textit{z}=0$.}\label{fig:Stars_AU24}
\end{figure}

\begin{figure}
    \centering
    \begin{subfigure}[b]{0.8\textwidth}
	    \includegraphics[width=\textwidth]{plots/Auriga/Au24_stars_xy_distribution_halo0_zoomin.png}
	    \label{fig:Au24_stars_xy_zoomin}
    \end{subfigure}
    
    \begin{subfigure}[b]{0.8\textwidth}
    \centering
    	\includegraphics[width=\textwidth]{plots/Auriga/Au24_stars_xz_distribution_halo0_zoomin.png}
    	\label{fig:Au24_stars_xz_zoomin}
    \end{subfigure}
    \caption{Stellar distribution of main galaxy at $\textit{z}=0$.}\label{fig:Stars_AU24}
\end{figure}

\subsection{Best fit potential}\label{subsec:best_fit_pot}
The best fit potential of a galaxy should be a analytic, axisymmetric potential so that we are able  to compare our results to the results of observers, since they mostly fit external galaxies in such a structure. We need to take into account that the galaxy evolved not isolated but went through many mergers and therefore its potential is either analytic nor axisymmetric but has a lot of substructure. The fit is only a vague approximation. Also it is self-consistent so changes in potential influence the velocities of the objects inside and changes on the positions of the objects will change the gravitational potential. Since we fit a potential to each snapshot we have a time-dependent potential. We also need to include a disk and bulge decomposition which is not natural in a cosmological simulation. As the gas evolves in cells, we cannot calculate its density easily. Therefore, we do not consider it in our potential fits. Since we want to recreate observers way of looking at galaxies we do not need to include gas as observers in e.g. \ac{MGE} fits \citep{MGE...Monnet, MGE...Emsellem} also only take stellar light into account. 

\subsubsection{Component decomposition}\label{subsubsec:decomp}
To fit a potential to each component, we first need to decompose the different parts. We assume that all \ac{DM} particles belonging to the main galaxy make up its halo. The stellar particles belong to either the spheroid or the disk. We distinguish these components by the use of the circularity parameter 
\begin{equation}
    \epsilon = \frac{L_z}{L_{z,max}(E)}
\end{equation}
where $L_{z,max}(E)$ is the maximum angular momentum allowed for the orbital energy $E$. 
$\epsilon = 1$ is a prograde circular orbit in the disc plane. $\epsilon = -1$ is a retrograde circular orbit in the disc plane. $\epsilon \sim 0$ is an orbit with a very low $z$-component of angular momentum which may be highly inclined to the disc spin axis and/or be highly eccentric.  

\cite{AurigaGrand} uses two different methods two distinguish the components and to get their mass ratio:
\begin{enumerate}
\item Under the assumption, that the bulge has zero net rotation, mirror negative $\epsilon$ as bulge material, the rest belongs to the disk.
\item All particles with $\epsilon > const$ are assigned to the disk, where $const = 0.7$ is set heuristically.
\end{enumerate}

\cite{AurigaGrand} find that the first method generally overestimates the disk-to-total ratio while the second approach underestimates it by choosing only kinematically very cold particles. Since we do not only want to get the mass ratio of disk-to-total but also want to tag each particle clearly, we use the second method. Nevertheless, the true assignment lies somewhere between these method. In Figure \ref{fig:decomposition}, we show a histogram of the circularity with our decomposition. The blue part is the disk portion while the yellow part is the spheroid. Together they add up to the black solid line, the total number. 
Since we can assign the particles to be in the spheroid or disk easily, we use this composition to find the disk and spheroid where we fit our potential to.

\begin{figure}
    \centering
    \includegraphics[width=0.7\textwidth]{plots/Auriga/decomposition_snap_127.png}
    \caption{Decomposition of stellar disk and spheroid by their kinematics. }
    \label{fig:decomposition}
\end{figure}
\subsubsection{Disk potential}\label{subsubsec:disk_pot}
We fit the disk with a \citet{MNprofile} potential following the profile 
\begin{equation}
\Phi(R,z) = -\frac{GM}{\sqrt{R^2+(a+\sqrt{z^2+b^2})^2}}
\end{equation} 
with scale length $a$ and scale height $b$ which provides a disk with a finite thickness. If $b\rightarrow 0$, the disk will be infinite thin and if $a \rightarrow 0$ the potential has a spherical density distribution. $b/a$ therefore defines the flattening of the system. It is a rather simply model with only a small computational costs. Therefore it is widely used. It has limitations in the mid-plane ($z=0$) at high $R$. 

\begin{figure}
    \centering
    \includegraphics[width = \textwidth]{plots/Auriga/MND_best_fit.png}
    \caption{2d density of MN disk data + best fit + absolute difference.}
    \label{fig:MND}
\end{figure}

\subsubsection{Spheroid potential}\label{subsubsec:spher_pot}
For the central stellar spheroid, we apply a \citet{Hernquistprofile} potential which has the density 
\begin{equation}
    \rho = \frac{M}{2\pi}\frac{a}{r}\frac{1}{(r+a)^3}
\end{equation}
where $M$ is the total stellar mass and $a$ is the scale length. 





\subsubsection{Halo Potential}\label{subsubsec:halo_pot}
We model the \ac{DM} halo with a \citet{NFWprofile} profile following the formula 
\begin{equation}
    \frac{\rho(r)}{\rho_{crit}} = \frac{\delta_c}{(r/r_s)(1+r/r_s)^2}
\end{equation} with the critical density $\rho_{crit} = 3H^2 / 8\pi G $, scale radius $r_s$  and a characteristic and dimensionless density $\delta_c$. The \ac{NFW} profile is derived from \ac{DM} only hydrodynamical simulations 






\subsubsection{Total potential}\label{subsubsec:tot_pot}
After fitting each component individually, we add them up to get a total potential. In Table \ref{tab:pot_best_fit_params}, we summarize our results for the last snapshot. \textbf{put in here a description of the potential and how each component contributes}
In Figure \ref{fig:pot_val_evol}, we show the time evolution of the potential parameters for the last \SI{10.5}{Gyr}. \textbf{here: describe evolution of potential and how mergers seem to influence them}

\begin{table}[htbp]
    \centering
    \begin{tabular}{@{}llll@{}}
         \toprule
         Component& Potential & Parameters &fitting method  \\
         \midrule
         stellar disk& Miyamoto-Nagai&\multirow[t]{3}{a_{\rm{MND}} = \SI{3}{kpc}\\b_{MND} = \SI{2}{kpc}\\v_0_{MN} = \SI{100}{km.s^{-1}}} & \multirow[t]{3}{\ac{MN} density fitted to density bins of disk in $(R,z)$.} \\
         & & & \\
         &&&\\
         \midrule
         stellar spheroid& Hernquist&\multirow[t]{2}{a_{\rm{HB}} = \SI{3}{kpc}\\v_0_{HB} = \SI{100}{km.s^{-1}}}& \multirow[t]{2}{Hernquist density fitted to density shells of spheroid in $(r)$.}\\
         &&&\\
         \midrule
         \ac{DM} halo&\ac{NFW}&\multirow[t]{2}{a_{\rm{NFWH}} = \SI{3}{kpc}\\v_0_{NFWH} = \SI{150}{km.s^{-1}}}&\multirow[t]{2}{Total potential fitted to 'pot' value of random DM particles where NFWH parameters where fitting parameters.}\\
         &&&\\
         \bottomrule 
    \end{tabular}
    \caption{Best fit potential overview: components, used potentials, their parameters and their fitting methods}
    \label{tab:pot_best_fit_params}
\end{table}



\begin{figure}
    \centering
    \begin{subfigure}[b]{0.3\textwidth}
	    \includegraphics[width=\textwidth]{plots/Auriga/surface_dens_disk_fit_data.png}
	    \label{fig:disk_surfdens_fit}
    \end{subfigure}
    ~ %add desired spacing between images, e. g. ~, \quad, \qquad, \hfill etc. 
      %(or a blank line to force the subfigure onto a new line)
    \begin{subfigure}[b]{0.3\textwidth}
    \centering
    	\includegraphics[width=\textwidth]{plots/Auriga/surface_dens_spher_fit_data.png}
    	\label{fig:spher_surfdens_fit}
    \end{subfigure}
    ~ %add desired spacing between images, e. g. ~, \quad, \qquad, \hfill etc. 
    %(or a blank line to force the subfigure onto a new line)
    \begin{subfigure}[b]{0.3\textwidth}
    \centering
    	\includegraphics[width=\textwidth]{plots/Auriga/surface_dens_halo_fit_data.png}
    	\label{fig:halo_surfdens_fit}
    \end{subfigure}
    
    \begin{subfigure}[b]{0.3\textwidth}
        \includegraphics[width=\textwidth]{plots/Auriga/volume_dens_disk_fit_data.png}
	    \label{fig:disk_voldens_fit}
    \end{subfigure}
    ~ %add desired spacing between images, e. g. ~, \quad, \qquad, \hfill etc. 
    %(or a blank line to force the subfigure onto a new line)
    \begin{subfigure}[b]{0.3\textwidth}
    \centering
    	\includegraphics[width=\textwidth]{plots/Auriga/volume_dens_bulge_fit_data.png}
    	\label{fig:spher_voldens_fit}
    \end{subfigure}
    ~ %add desired spacing between images, e. g. ~, \quad, \qquad, \hfill etc. 
    %(or a blank line to force the subfigure onto a new line)
    \begin{subfigure}[b]{0.3\textwidth}
        \centering
    	\includegraphics[width=\textwidth]{plots/Auriga/volume_dens_halo_fit_data.png}
	    \label{fig:halo_voldens_fit}
    \end{subfigure}
    \caption{Surface densities (upper row) and mass densities (lower row) of the components and their best fits at $\textit{z}=0$. Left (blue): stellar disk, middle (green): stellar spheroid, right (yellow): \ac{DM} halo.}\label{fig:single_pot_fits}
\end{figure}

\begin{figure}[htbp]
\centering
	\includegraphics[width=0.5\textwidth]{plots/Auriga/circ_vel_fit_data_test.png}
	\caption{Circular velocity at \textit{z} = 0: data, total, disk, bulge and halo..}
	\label{fig:circ_vel_fit}
\end{figure}

\begin{figure}[htbp]
\centering
	\includegraphics[width=0.9\textwidth]{plots/Auriga/fitted_potential_evolution_dec18.png}
	\caption{Evolution of all fit parameters over time.}
	\label{fig:pot_val_evol}
\end{figure}

\subsection{What not to do when fitting a gravitational potential}\label{subsec:wrong_pot_fit}







\section{Adaptive dynamics: Can actions of accreted globular clusters constrain the gravitational potential?}
\subsection{Integrals of motion}

\subsubsection{Energy and angular momentum}
\subsubsection{Actions}
\begin{itemize}
    \item General action introduction (heuristic / properly)
    \item explain Staeckel fudge
    \item mention galpy
    
\end{itemize}

\subsubsection{Excursion: coordinate transformations}
private communication with Wilma Trick
\subsection{Merger tree}
We say th
figure: Merger tree of Auriga 24. 
\subsection{Globular cluster sample selection}
Due to the resolution of the simulation, $\rm{M} = 5 \cdot 10 ^ 4\ M_{\odot}$, we set one stellar particle as one \ac{GC}. We find the three biggest \ac{DG} merger events in the merger tree. All stellar particles which were accreted by the main halo are followed through the evolution and kept as accreted \acp{GC} as long as they do not cross the disk in a sense that they either are directly in the disk - defined per snapshot as within the radius $\rm{R}_d \<= 0.1 \ R_{200}$ and the height $\rm{z}_d = 0.03\ R_d$ to match the \ac{MW} disk's height in the $z = 0$ snapshot - or changed signs between successive snapshots while being inside the disk radius since we assume that in these cases the \ac{GC} would be disrupted. 


\begin{table}[htbp]
    \centering
    \begin{tabular}{@{}lllll@{}}
        \toprule
         \makecell[l]{name}& \makecell[l]{merger time}& \makecell[l]{mass of \ac{DG}\\ at merger} & \makecell[l]{num of \\accreted particles} & \makecell[l]{mass of \\accreted particles}\\
         \midrule
         prog2& & && &\\
         prog3& & &&&\\
         prog4& & &&&
         \bottomrule
    \end{tabular}
    \caption{Progenitor parameters}
    \label{tab:prog_overview}
\end{table}


\begin{figure}[htbp]
    \centering
    \begin{subfigure}[c]{0.45\textwidth}
    \centering
    	\includegraphics[width=\textwidth]{plots/Dynamics/dist/xy_dist_selected_GCs_prog_2_snap_127.png}
    	\label{fig:prog2_xy}
    \end{subfigure}
    ~ %add desired spacing between images, e. g. ~, \quad, \qquad, \hfill etc. 
    %(or a blank line to force the subfigure onto a new line)
    \begin{subfigure}[c]{0.45\textwidth}
        \centering
    	\includegraphics[width=\textwidth]{plots/Dynamics/dist/xz_dist_selected_GCs_prog_2_snap_127.png}
	    \label{fig:prog2_xz}
    \end{subfigure}
    
    \begin{subfigure}[c]{0.45\textwidth}
    \centering
    	\includegraphics[width=\textwidth]{plots/Dynamics/dist/xy_dist_selected_GCs_prog_3_snap_127.png}
    	\label{fig:prog3_xy}
    \end{subfigure}
    ~ %add desired spacing between images, e. g. ~, \quad, \qquad, \hfill etc. 
    %(or a blank line to force the subfigure onto a new line)
    \begin{subfigure}[c]{0.45\textwidth}
        \centering
    	\includegraphics[width=\textwidth]{plots/Dynamics/dist/xz_dist_selected_GCs_prog_3_snap_127.png}
	    \label{fig:prog3_xz}
    \end{subfigure}
    
    \begin{subfigure}[c]{0.45\textwidth}
    \centering
    	\includegraphics[width=\textwidth]{plots/Dynamics/dist/xy_dist_selected_GCs_prog_4_snap_127.png}
    	\label{fig:prog4_xy}
    \end{subfigure}
    ~ %add desired spacing between images, e. g. ~, \quad, \qquad, \hfill etc. 
    %(or a blank line to force the subfigure onto a new line)
    \begin{subfigure}[c]{0.45\textwidth}
        \centering
    	\includegraphics[width=\textwidth]{plots/Dynamics/dist/xz_dist_selected_GCs_prog_4_snap_127.png}
	    \label{fig:prog4_xz}
    \end{subfigure}
    \caption{.}\label{fig:progenitors_distribution}
\end{figure}


\subsection{Globular clusters in action space}\label{subsec:GCs_action_space}
Now, we look at the \ac{GC} distribution in action space. Our assumption is that in the "true" potential, \acp{GC} are very clumped since they should retain dynamical memory from their former \ac{DG} and therefore their \ac{DF} should be a delta function. In section \ref{subsubsec:GCs_actions_right_pot}, we will look at the distribution in the fitted potential at redshift 0. In section \ref{subsubsec:GCs_actions_varying_pot}, we evaluate actions in varying potentials to test our assumption of \acp{GC} being most clumped in action space in the "true" potential. 

\subsubsection{Best fit potential}\label{subsubsec:GCs_actions_right_pot}

\begin{figure}[htbp]
    \centering
    \includegraphics[width=1.0\textwidth]{plots/Dynamics/prog234_actions_snap_127.png}
    \caption{Selected \acp{GC} from three different \acp{DG} in action space.}
    \label{fig:act_both_merg_best_pot}
\end{figure}



\subsubsection{Varying potentials}\label{subsubsec:GCs_actions_varying_pot}

\begin{figure}[htbp]
    \centering
	\includegraphics[width=0.6\textwidth]{plots/Dynamics/prog2/a_NFW_diagnostic_plot_std_prog2_all.png}
	\label{fig:NFW_diag_prog2}
\caption{.}
\end{figure}

\begin{figure}[b]
    \centering
	\includegraphics[width=0.6\textwidth]{plots/Dynamics/prog3/a_NFW_diagnostic_plot_std_prog3_all.png}
    \label{fig:NFW_diag_prog3}
\caption{.}
\end{figure}
\begin{figure}[b]
    \centering
	\includegraphics[width=0.6\textwidth]{plots/Dynamics/prog4/a_NFW_diagnostic_plot_std_prog4_all.png}
    \label{fig:NFW_diag_prog4}
\caption{.}
\end{figure}

\subsection{Time evolution of actions}\ref{subsec:}
We evaluate the time evolution of the orbits of the accreted \acp{GC} to how the actions evolved and to see if there was a point - probably shortly after their mergers - where the \acp{GC} were more clumped in action space. We calculate the actions of the selected particles in the best fit potential in each snapshot.  

\begin{figure}[htbp]
    \centering
	\includegraphics[width=\textwidth]{plots/Dynamics/prog2/action_time_evolution_hist_mean.png}
	\label{fig:time_ev_all_GCs}
	\caption{.}
\end{figure}
\begin{figure}
    \centering
	\includegraphics[width=\textwidth]{plots/Dynamics/prog3/action_time_evolution_hist_mean.png}
    \label{fig:time_ev_box_GCs}
    \caption{.}\label{fig:actions_time_evolution}
\end{figure}
\begin{figure}
    \centering
	\includegraphics[width=\textwidth]{plots/Dynamics/prog4/action_time_evolution_hist_mean.png}
    \label{fig:time_ev_box_GCs}
    \caption{.}\label{fig:actions_time_evolution}
\end{figure}


\section{Discussion} \label{sec:Discussion}
\subsection{Comparison to observations}
\textit{Gaia}-Enceladus \cite{Enceladus....Helmi...2018}
sagitarius modellierung single orbit Beispiel
-> zeigen dass kein single orbit ist (literatursuche)

\begin{figure}[htbp]
    \centering
    \includegraphics[width=1.0\textwidth]{plots/Discussion/Gaia_all_actions_MW14_talk3.png}
    \caption{\textit{Gaia} Enceladus in action space in the MW14Potential.}
    \label{fig:act_both_merg_best_pot}
\end{figure}

\subsection{Comparison to literature}
richtige verteilunggsfunktion fuer accreted teilchen zu bestimmen 

\subsection{Caveats}

\subsection{Future Work}


\newpage
\section*{Acronyms}
\begin{acronym}[NFW]
    \acro{AGN}{active galactic nuclei}
    \acro{DG}{dwarf galaxy}
	\acro{DM}{dark matter}
	\acro{GC}{globular cluster}
	\acro{IFU}{Integral Field Unit}
	\acro{MGE}{Multi-Gaussian Expansion}
	\acro{MHD}{magnetohydrodynamical}
	\acro{MW}{Milky Way}
	\acro{NFW}{Navarro-Frenk-White}

\end{acronym}
\newpage
\bibliography{mybib}{}
\bibliographystyle{mnras}
\end{document}
